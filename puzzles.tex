\section{Puzzles}

\begin{figure}[H]
  \centering
  \chessboard[moverstyle=circle, mover=w, setwhite={Ke1, Ra1, Nf1, Be2, Qh8, Pa2, Pb2, Pc2, Pf2, Pg3}, addblack={Kf8, Qh1, Re8, Ng8, Bg4, Pa7, Pb7, Pc7, Pd6, Pf7, Ph7}]
  \caption{What move avoids a forced checkmate?}
  \label{fig:puzzle-1}
\end{figure}

\begin{figure}[H]
  \centering
  \chessboard[moverstyle=circle, mover=w, setwhite={Kg1, Rf1, Ra1, Qb6, Bc4, Pb3, Pd3, Pf2, Pg2, Ph2}, addblack={Kg8, Rd4, Qg5, Pf7, Pg6, Ph7, Nh6}]
  \caption{What move initiates an +M1 or captures a queen?}
  \label{fig:puzzle-2}
\end{figure}

\newpage
\subsection{Solutions}

{{\hypersetup{hidelinks}\paragraph{\hyperref[fig:puzzle-1]{Puzzle 1}} \textbf{1. O-O-O}}

\begin{figure}[H]
  \centering
  \chessboard[
    moverstyle=circle,
    mover=w,
    setwhite={Ke1, Ra1, Nf1, Be2, Qh8, Pa2, Pb2, Pc2, Pf2, Pg3},
    addblack={Kf8, Qh1, Re8, Ng8, Bg4, Pa7, Pb7, Pc7, Pd6, Pf7, Ph7},
    pgfstyle=straightmove,
    color=red,
    markmoves={a1-d1, e1-d1, e1-d2, a2-a3, a2-a4, b2-b3, b2-b4, c2-c3, c2-c4, h8-h7, h8-g7, h8-g8, h8-f6, h8-e5, h8-d4, h8-c3, f2-f3, f2-f4, a1-b1, a1-c1},
    color=green,
    markmoves={e1-c1}
  ]
  \caption{The only good move is \textbf{1. O-O-O}. All other moves are marked in red.}
  \label{fig:puzzle-sol-1}
\end{figure}

The following list discusses all possible moves and shows why they lead to mate.

\begin{itemize}
  \item \textbf{1. \symqueen xh7} looses a queen.
  \item \textbf{1. \symqueen xg8+}, \textbf{1. \symqueen g7+} and \textbf{1. \symqueen e5} looses a queen and doesn't prevent the forced checkmate (see point 4).
  \item \textbf{1. f3} is pointless, because of \textbf{1... \symbishop xf3}.
  \item \textbf{1. [a/b/c][3/4]}, \textbf{1. f4} and \textbf{1. \symrook [b/c]1} allow -M2 with \textbf{1... \symrook xe2+ 2. \symking d1 \symqueen xf1\#}.
  \item \textbf{1. \symrook d1} allows -M1 with \textbf{1... \symrook xe2\#}.
  \item \textbf{1. \symking d1} allows -M2 with \textbf{1... \symrook xe2 2. ? \symrook xf1\#}. White's second move cannot prevent the mate.
  \item \textbf{1. \symking d2} allows -M2 with \textbf{1... \symrook xe2+ 2. \symking [c/d]1 \symqueen xf1\#} and -M5 with \textbf{1... \symrook xe2+ 2. \symking [c/d]3 \symqueen f3+ 3. \symking [b/c/d]4 \symrook e4+ 4. \symking [a/b/d]5 \symqueen f5+ 5. \symqueen e5 \symqueen xe5\#}
\end{itemize}


\newpage
{{\hypersetup{hidelinks}\paragraph{\hyperref[fig:puzzle-2]{Puzzle 2}} \textbf{1. Qxd4}}

\begin{figure}[H]
  \centering
  \chessboard[
    moverstyle=circle,
    mover=w,
    setwhite={Kg1, Rf1, Ra1, Qb6, Bc4, Pb3, Pd3, Pf2, Pg2, Ph2},
    addblack={Kg8, Rd4, Qg5, Pf7, Pg6, Ph7, Nh6},
    pgfstyle=straightmove,
    color=green,
    markmoves={b6-d4},
  ]
  \caption{The best move is \textbf{1. Qxd4}. Possible follow-ups and white's response are marked with arrows.}
  \label{fig:puzzle-sol-2}
\end{figure}

The following list discusses black's possible responses to white's mate threat and details white's best response.

\begin{enumerate}
  \item \textbf{1... \symqueen [a5/e5/f6/xg2+]} looses a queen and can't prevent the checkmate. \textbf{2. [\symrook xa5/\symqueen xe5/xf6/\symking xg2] ... 3. \symrook a8\#}. Black has no possible counter for the checkmate threat as move 2.
  \item \textbf{1... [\symknight[f5/g4]/\symqueen [e7/c1/d2/g2/e3/g3/[f/g/h]4/[b/c/d/f/h]5]]} does nothing. \textbf{2. \symrook a8\#}
  \item \textbf{1... \symqueen d8} delays the mate to +M5. \textbf{2. \symqueen xd8+ \symking g7 3. \symqueen d4+ \symking f8 4. \symrook a8+ \symking e7 5. \symqueen d8\#}
\end{enumerate}